% !TEX TS-program = xelatex
% !TEX encoding = UTF-8 Unicode
% !Mode:: "TeX:UTF-8"

\documentclass{resume}
\usepackage{zh_CN-Adobefonts_external} % Simplified Chinese Support using external fonts (./fonts/zh_CN-Adobe/)
%\usepackage{zh_CN-Adobefonts_internal} % Simplified Chinese Support using system fonts
\usepackage{linespacing_fix} % disable extra space before next section
\usepackage{cite}

\begin{document}
\pagenumbering{gobble} % suppress displaying page number

\name{郭嘉庆}

\basicInfo{
  \email{648591658@qq.com} \textperiodcentered\ 
  \phone{(+86) 13669223711}
} 
\section{\faGraduationCap\  教育背景}
\datedsubsection{\textbf{清华大学}, 北京}{2019年8月 -- 2022年7月}
\textit{硕士研究生}\ 电气工程及其自动化
\datedsubsection{\textbf{西安交通大学}, 西安, 陕西}{2015年8月 -- 2019年7月}
\textit{本科}\ 电气工程及其自动化

\section{\faUsers\ 项目经历}
\datedsubsection{\textbf{华为} ~~深圳 ~~\textit{开发工程师A}}{}
\datedsubsection{\textbf{RAG问答系统开发}}{2025年1月 -- 至今}
\begin{onehalfspacing}
主导Deveco Studio开发者智能问答系统建设,基于RAG架构实现技术文档智能检索与生成,内部日均使用人次300+, 问题解决反馈占比25\%
\begin{itemize}
  \item 技术文档递归分块:针对开发文档、需求说明书文档层级结构特性,采用层级递归的分块策略,通过BAAI/bge模型对分块进行向量编码,存入Milvus向量数据库(Recall@5为81.2\%)
  \item 搭建混合召回引擎:融合BM25关键词检索与BAAI/bge-me语义检索,实现多路召回结果加权融合,在400条集上recall@5提升至86.7\%
  \item 优化生成模型效果:基于历史3000+条高质量QA数据微调Qwen-32B,采用LoRA轻量化训练策略,自动化评测显示回答准确率由68\%提升至79\%
\end{itemize}

\datedsubsection{\textbf{DevEco Studio 调试开发}}{2022年10月 -- 2024年12月}
\begin{onehalfspacing}
负责DevEco Studio 调试团队管理运作以及需求交付
\begin{itemize}
  \item 对调试启动流程进行优化,对断点进行批处理,sourcemap进行按需加载,实现调试启动速度提升90\%(3000+文件工程调试启动时间由30s降低至3s), 内存占用降低30\%(2GB -> 1.4GB)
  \item 完成ArkTs与Cpp跨语言调试方案设计,实现ArkTs与Cpp跨语言堆栈融合以及无缝单步调试,方案申请专利一项
  \item 设计开发release堆栈反混淆工具,支持一键、批量解析release应用日志,提升开发者日志定位效率,用户使用率占DevEco Studio 用户总数85\%
\end{itemize}
\end{onehalfspacing}

\datedsubsection{\textbf{360集团} ~~北京 ~~\textit{推荐系统实习生}}{}
\datedsubsection{\textbf{推荐系统开发}}{2022年10月 -- 2024年12月}
\begin{onehalfspacing}
基于RALM模型,推荐系统中中长尾内容的相似拓展效果
\begin{itemize}
  \item 通过注意力机制,对用户多维度特征进行合并,提高用户表示的有效性
  \item 通过全局注意力以及局部注意力机制对种子用户表示进行优化,提升种子用户作用于不同目标用户的有效性和鲁棒性
  \item 相比于baseline(LR),RALM在离线测试集中,AUC提升类6.6\%, prec@10提升了3.1\%
\end{itemize}
\end{onehalfspacing}

% Reference Test
%\datedsubsection{\textbf{Paper Title\cite{zaharia2012resilient}}}{May. 2015}
%An xxx optimized for xxx\cite{verma2015large}
%\begin{itemize}
%  \item main contribution
%\end{itemize}

\section{\faInfo\ 其他}
% increase linespacing [parsep=0.5ex]
\begin{itemize}[parsep=0.5ex]
  \item \datedsubsection{终端云服务总裁奖}{2024年1月}
  \item \datedsubsection{思源奖学金}{2017年8月}
  \item 英语-熟练
\end{itemize}

%% Reference
%\newpage
%\bibliographystyle{IEEETran}
%\bibliography{mycite}
\end{document}
